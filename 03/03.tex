% Vorlesung vom 16-11-08
\begin{figure}[H]
	\centering
	\begin{tikzpicture}[scale=4, every node/.style={inner sep = 0, outer sep = 0}]
		\draw[-latex] (0,0)--(2,0) node[pos=.9, below] {$e_1 = \vektor{1\\0}$};
		\draw[-latex] (0,0)--(0,1.5);
		\draw[-latex] (0,0)--(1,.5) node (A) {};
		\node[fill, circle, inner sep = 1.5, label={[label distance = .2cm]below:$b_0$}] at (A) {};
		\draw[dashed] (A) -- (2,.5) node (B) {};
		\draw[-latex] (A) -- ($ (A)+(.866, .5) $) node (C) {};
		\draw[-latex] (A) -- ($ (A)+(-.5, .866) $) node (E) {};
		\node[label = {[label distance=.15cm]190:{$\{B\}$}} ] at (E) {};
		\node[label = {[label distance=.15cm]0:{$ b_2 = \vektor{-\sin \varphi\\ \cos \varphi} $}} ] at (E) {};
		\node[label = {[label distance=.15cm]0:{$ b_1 = \vektor{\cos \varphi\\ \sin \varphi} $}} ] at (C) {};
		\node[fill, circle, inner sep = 1, label={0:${}^B x = \vektor{B_{x_1}\\B_{x_2} }$}] at (1.5,.25) {};
		\node[label ={[label distance=.5cm] 5:$\varphi$}] at (A) {};
		\pic[draw=black,angle eccentricity=1.2,angle radius=1cm] {angle=B--A--C};
		\draw[very thin] (C)--($ (A)+(.866,0) $) node (D) {};
		\node[label = 135:$\cdot$] at (D) {};
		\pic[draw=black, angle eccentricity=1.2,angle radius=.25cm] {angle=C--D--A};
		\node[label={[label distance=.15cm]180:{$\{E\}$}}] at (0,1.5) {};
		\node[label={[label distance=.15cm]0:{$ e_2 = \vektor{0\\1} $}}] at (0,1.5) {};
	\end{tikzpicture}
\end{figure}
\begin{description}
	\item[$b_0$:] E-Koordinaten des Ursprungs von System B
	\item[$ b_1, b_2 $] E-Koordinaten der Basisvektoren von System B
\end{description}
\[ {}^Ex = b_0 + {}^Bx_1 \cdot b_1 + {}^Bx_2 \cdot b_2 \]
\[ [b_1,b_2] = R \in \mathbb{R}^{2\times2},~~\left| b_1 \right| = \left| b_2 \right| = 1, ~~ \overset{\text{Standardskalarprodukt}}{\overbrace{b_1^T\cdot b-2}} = 0 \]
\[ R^T \cdot R = \mathbb{E}, \det(R) = 1 \text{ (Rechtssystem)} \]
\[ \Rightarrow {}^Ex = b_0 + {}^ER\cdot{}^Bx \]
\section{Homogene Koordinaten}
\[{}^Ex^1 = \vektor{{}^Ex_1\\{}^Ex_2\\1} = \vektor{{}^Ex\\1} \in \mathbb{R}^3 \]
\[ = \left( \begin{array}{@{}c|c@{}}
{}^ER_B&b_0\\ \hline0&1
\end{array} \right) \vektor{B_x\\1} \]
\[ \vektor{\cos\varphi&-\sin\varphi&b_{0_1} \\ \sin\varphi&\cos\varphi&b_{0_1}\\0&0&1} \vektor{{}^Bx_1\\ {}^Bx_2 \\ 1  } \]
\paragraph{Allgemein: }
\[ {}^Ex^1 = {}^EM_B \cdot {}^B \hat{x} \]
\begin{enumerate}
	\item ${}^EM_B$ beschreibt die Transformationsmatrix von Koordinaten aus System B in das System E
	\item ${}^EM_B$ kann auch interpretiert werden, als die (starre) Transformation, die E in B überführt.
\end{enumerate}
\section{Transformationen}
\subsection{Rotation}
\paragraph{z.B.}
$~$\\
\begin{figure}[H]
	\centering
	\begin{tikzpicture}[scale=4]
		\draw[-latex] (0,0)--(1,0) node[below, pos=.9] {$e_2$};
		\draw[-latex] (0,0)--(0,1) node[left, anchor = east, pos=.9] {$e_1$};
		\draw[-latex] (0,0)--(.5,.866) node[above] {$x'$};
		\draw[-latex] (0,0)--(.866,.5) node[above] {$x$};
		\node at (.5,.866) (x') {};
		\node at (.866,.5) (x) {};
		\node[label = 45:$\varphi$ ] at (0,0) (0) {};
		\pic[draw=black,angle eccentricity=1.2,angle radius=1cm] {angle=x--0--x'};
	\end{tikzpicture}
\end{figure}
\[ x'=R\cdot x \]
\subsection{+ Verschiebung}
\[ x' = Rx + z \]
$\rightsquigarrow \hat{x}' = Mx$ mit $M = \left( \begin{array}{@{}c|c@{}} R&t\\ \hline 0&1\end{array} \right)$
\[ x' = R(x + t) \]
\subsection{Skalierung}
\[ \underset{\hat{S} = \overset{\text{homogenisiert}}{\left( \begin{array}{@{}c|c@{}} S&0\\ \hline0&1 \end{array} \right)}}{\underbrace{S = \vektor{s_1&0\\0&s_2}}} ~~ x'=Sx = \vektor{x_1' = s_1x_1 \\ x_2' = s_2x_2} \]
mit $s_2 = -1$ Spiegelung um $x_1$ % Konnte ich nicht lesen
\subsection{Translation}
\paragraph{Homogen:}
\[ \hat{T} = \left( \begin{array}{@{}cc|c@{}}
1&0&t_1\\0&1&t_2\\ \hline 0&0&1
\end{array} \right) \]
\pagebreak
\subsection{Hintereinanderausführung von Translation, Rotation und Skalierung}
\paragraph{z.B.}
\[ x''' = S(R(x+t)) \]
\[ \rightsquigarrow ~x' = x+t;~~x''=Rx';~~x'''=Sx'' \]
\[ x''' = \underset{\overleftarrow{\text{Leserichtung}}}{\hat{S}\hat{R}\hat{T}\hat{x}}~~~\text{alles homogenisiert} \]
$\hat{}$ wird oft weggelassen, wenn klar.
\begin{figure}[H]
	\begin{subfigure}{.3\textwidth}
		\centering
		\begin{tikzpicture}[scale=2]
		\draw[-latex] (0,0) -- (0,1) node[pos=.9, left] () {$\{E\}$};
		\draw[-latex] (0,0) -- (1,0);
		\node[fill, circle, inner sep = .5, label=${}^Ex$] at (.5,.5) {};
		\end{tikzpicture}
		\caption{Ausgang}
	\end{subfigure}
	$\leftrightarrow$
	\begin{subfigure}{.3\textwidth}
		\centering
		\begin{tikzpicture}[scale=2]
		\draw[-latex] (0,0) -- ($({-sin(25)}, {cos(25)})$) node[pos=.9,below left] () {$\{B\}$};
		\draw[-latex] (0,0) -- ($({cos(25)}, {sin(25)})$);
		%\node[fill, circle, inner sep = .5] at ($({.5*cos(25) - .5*sin(25)},{.5*sin(25) + .5*cos(25)})$) {};
		\node[fill, circle, inner sep = .5, label=${}^Bx$] at (.5,.5) {};
		\end{tikzpicture}
		\caption{Rotation um $25^\circ$}
	\end{subfigure}
	$\leftrightarrow$
	\begin{subfigure}{.3\textwidth}
		\centering
		\begin{tikzpicture}[scale=2]
		\draw[-latex] (-.2,-.1) -- ($(-.2,-.1) + ({-sin(25)},{cos(25)})$) node[pos=.9, below left] () {$\{A\}$};
		\draw[-latex] (-.2,-.1) -- ($(-.2,-.1) + ({cos(25)},{sin(25)})$);
		\node[fill, circle, inner sep = .5, label=${}^Ax$] at (.5,.5) {};
		\end{tikzpicture}
		\caption{Translation um $\vektor{-0,2\\-0,1}$}
	\end{subfigure}
\end{figure}
\[ {}^BM_A{}^Ax = {}^Bx \]
\[ {}^Ex = {}^EM_B{}^Bx \]
\[ \Rightarrow {}^Ex = \underset{{}^EM_A}{\underbrace{{}^EM_B{}^BM_A}}{}^Ax \]
\paragraph{z.B.}
\subparagraph{gegeben}
\[ {}^EM_B~~~{}^EM_A \]
\subparagraph{gesucht}
\[ {}^BM_A \]

\[ {}^EM_A = {}^EM_B{}^BM_A \]
\[ \Rightarrow {}^BM_A = {}^EM_B^{-1}\cdot{}^EM_A \]
\[ {}^EM_B^{-1} = {}^BM_E \]
\section{Invertierung von $M$}
Sei $M = \left( \begin{array}{@{}c|c@{}}
R&t\\ \hline 0 & 1
\end{array} \right)$ $ ~~~~~\underset{\overset{\uparrow}{\text{Drehung um }-\varphi}}{R^{-1}} = R^T$ u.a. auch, da Rotation $(R^TR = 1)$ % Richtig so ?
\[ Mx = Rx + t = x' \]
\[ \rightsquigarrow R^T (x'-t) = x \]
\[ \rightsquigarrow R^Tx' - R^Tt \]
\[ \rightsquigarrow M^{-1} = \left( \begin{array}{@{}c|c@{}}
R^T&-R^Tt\\ \hline 0 & 1
\end{array} \right) \]
\section{Qt}
\begin{lstlisting}[language=c++]
QMatrix4x4 M;
M.setToIdentity();		// $M = \mathds{1}$
M.rotate($\varphi$, 0, 0, 1);	// Rotation um die z-Achse
						// $M = M \cdot R$
M.scale($s_1$, $s_2$);			// $M = M \cdot S$
M.translate($t_1$, $t_2$);		// $M = M \cdot T$
// $M = \underset{\overleftarrow{\text{Leserichtung für Transformation}}}{\mathds{1}\cdot R\cdot S \cdot T}$
// Leter Befehl wird zuerst ausgeführt! LIFO!
Mx'
\end{lstlisting}